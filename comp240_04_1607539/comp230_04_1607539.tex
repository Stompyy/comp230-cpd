% Please do not change the document class
\documentclass{scrartcl}

% Please do not change these packages
\usepackage[hidelinks]{hyperref}
\usepackage[none]{hyphenat}
\usepackage{setspace}
\doublespace

% You may add additional packages here
\usepackage{amsmath}

% Please include a clear, concise, and descriptive title
\title{CPD Report}

% Please do not change the subtitle
\subtitle{COMP240 - CPD Report}

% Please put your student number in the author field
\author{Student Number: 1607539}

\begin{document}

\maketitle

\section*{Introduction}

As the second year of this course draws to a close, I find myself thinking on what will come after this degree. My future career that stems from this qualification is the sole reason for my studies. Luckily, the broad coverage of this degree offers many possible routes within computing. An early decision will afford more time to better prepare, and with a demanding year ahead, I cannot linger on my choice of path for too long.
 
\section*{Career path}

Cornwall does not have a large amount of game development jobs. This country does not have a large amount of game development jobs. It would be foolish to restrict myself to such a small niche, especially considering that so many of the skills that I am learning are applicable to so many other areas of computing. The problem that I have is in narrowing such a wide field down into a set of choices that I can explore further. Reflecting on my previous personal development reports, the office visits and networking have helped me to form opinions on very small areas of software development but this must vastly increase before I have enough information to inform my decision. \textit{`Software Cornwall'} is a good candidate to help me with this, so I intend on becoming involved in some way by attending an upcoming event or tech jam \href{https://www.softwarecornwall.org/events/}{(\textit{link})} over the summer. Michael has suggested this before, so I should speak with him to learn more. Previous plans to network further by myself have fallen by the wayside due to workload and organisation, but over the summer there will be no immediate deadlines and these events are designed for such, so this goal is reasonably easily attained.

\section*{Other than game development?}

So given that game development jobs are in relative short supply currently in Cornwall what else is there? My passion for game development is rooted firmly in the act of creating. To work on a project and have something tangible at the end, as opposed to maintenance or service provided. Something that has value to someone other than myself and can be sold. Exploring this avenue away from game development, an obvious example would be tool creation. I have enjoyed great satisfaction with this from my python map reader program in the first year. To grow my experience and `test the water' a little more I aim to make at least one tool over the summer before the third-year lectures begin, that I feel has worth as opposed to purely as an exercise. Due to ease and current success using \textit{winforms}, I will likely use this again, although I have also researched browser plug-ins. Releasing to market will be a true measure of success, although as with previous endeavours released without marketing, results can be minimal.

\section*{Portfolio}

Despite a pragmatic approach to employment likelihood, I am not ready to give up on game development just yet. It was my hobby long before attending university and I will undoubtably be making games long after. With advice from Gareth regarding how to further my game developer portfolio, I have started a Github repository solely to create and recreate many small gameplay slices that interest me. I aim to fill it with many small games or mechanics using any suitable software or language that is appropriate, or again, any that interest me. Recreating a \textit{Monte Carlo Tree Search} using Unreal Engine's blueprint visual scripting is an example. As an achievable target and a decent start, I aim to produce five such small projects before September. This is not to give myself extra work but to give my personal projects and experiments some direction and perhaps greater value as a cohesive whole.

I have also invited others from the group as collaborators to spur each other on and promote a group project approach that will hopefully be beneficial to all involved.

\section*{Pocket pals}

Due to some limited experience developing for mobile devices, Tristan Barlow and I have been recruited to work on developing \textit{Pocket Pals} for mobile devices over the summer. This is a good opportunity to work on a wholesome game idea as opposed to the shooting and fighting games that we often develop. But the nature of this project and the time frame will offer many new problems. Firstly this a real paid job offer. Instead of the drop in, drop out casual approach to \textit{Battle Screens'} development, real hours must be logged and external expectations and targets will be set. Communication will be key, and I am not confident that a purely remote development environment will be sustainable. Given our previous experience, Tristan and I can consistently work effectively with each other. But, given that this new team have no development experience or experience with us, great care must be taken to communicate effectively. I propose a fortnightly meeting in person, in addition to the remote daily stand ups and weekly sprint reviews. The distance between each other may be problematic but the final product will be the measure of success.

\section*{Launchpad}

Foremost in my mind, and the underlying direction of all previous targets, is the wish to gain a place on Falmouth University's \textit{Launchpad} master's program. This was suggested as being suitable for me after a previous CPD report. My plan has always been to complete a master's after this degree, but there is a limited choice of appropriate MSc computing courses in Devon and Cornwall. Despite not being an MSc, the Launchpad program offers an excellent complement to a BSc in computing and supplies real opportunities. I have approximately ten months to carve myself into the best possible candidate which will involve each of the above targets. Further to this, I need more information. I am trying to stay abreast of the application procedure as a computing student from a third-year applicant James. He is helpful, but I must be persistent in following his progress, specifically the intimidating code portion of the interview. I must be fully informed and prepared before applying.

Lastly, I have spoken to Phoebe, a Launchpad graduate who has achieved an enviable position by using the Launchpad program to stay within the university's infrastructure. This is a fantastic incentive if this is indeed a possible outcome.

\section*{Conclusion}

I am (perhaps overly) confident in my abilities and attitude so those subjects are not apparent here to fix by actions or setting targets. Instead I am concerned with learning from what I have enjoyed and what is realistic, and setting my personal targets accordingly. I tend to over think decisions, but this is prudent when they affect my future career.

\end{document}
\end{document}
