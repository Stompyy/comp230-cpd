% Please do not change the document class
\documentclass{scrartcl}

% Please do not change these packages
\usepackage[hidelinks]{hyperref}
\usepackage[none]{hyphenat}
\usepackage{setspace}
\doublespace

% You may add additional packages here
\usepackage{amsmath}

% Please include a clear, concise, and descriptive title
\title{CPD Report}

% Please do not change the subtitle
\subtitle{COMP230 - CPD Report}

% Please put your student number in the author field
\author{Student Number: 1607539}

\begin{document}

\maketitle

\section*{Introduction}

Unsurprisingly, this has been a far more eventful year than the first. With module grades now contributing to the final grade, the pressure to do well has been mounting as well as the difficulty of the tasks. With a mind to specialise as soon as possible, I have spent considerable time on mobile development. These small scale projects suit my individual learning needs and enable me to release market ready products without a large team. I aim to fill my portfolio with many of these small projects, as a demonstration of my abilities, and to gain experience in taking products to market.
 
\section*{Other commitments}

My main objective for returning to education is to enable a career change. A hospitality job entails working long unsocial hours for little more than minimum wage, and it is with cold irony that with the decision to finally change career comes at least another three years of doing the exact same. Whilst I have shed some responsibilities, I still must work and travel an estimated forty hours per week. This must be carefully scheduled around lectures and other commitments, and is taking many different attempts to perfect. Currently I leave my free evenings to do university work until the end of the week, but this work can easily start to pile up. There is no easy solution to this. Going forward, I will next try working alternate days, and I have been given permission by my employer to change my days at will. I have until the end of this second year term to get this balance right before the third year begins, when the workload will increase yet again.

Additionally, although perhaps coincidentally, after requesting to be able to park closer to university, this has now been allowed. This has saved me nearly four hours a week (half an hour each day) walking time, which while not life changing, is a start to freeing up a very constricted schedule.

\section*{Portfolio}

With the exception of a degree, I am of the opinion that experience is worth more than qualifications in real world employment opportunities. A degree is often a necessary prerequisite to a job application, but it is the type and quantity of experience that will place me above other graduates. To this end, and given that I hold a distribution license for iOS and android platforms, I have approached my university's module work with a view to hopefully take some to market. Ambitious, but also possible. This was recently achieved with the VR/AR module, gifting the world ``Paper AR Plane'' for iOS! A very different project to ``Battle Screens'' which I and others released in August 2017. With the focus on filling my portfolio rather than making money, I set the price to be free, and have felt strangely liberated by this. I plan on releasing many more small projects as I progress, likely with the same focus on portfolio expansion rather than making money. With an artificial intelligence module and a distributed systems module fast approaching, my aim is to develop one of these projects into another finished and polished product, and release it before my distribution licenses expire in July.

\section*{Brushing up}

Despite the course content being very broad at Falmouth University, learning languages from visual Unreal Engine blueprints, to C++, there exists so many languages that there is no hope of covering a significant portion. Also, some that we might glance over may not be given an appropriate amount of attention to learn sufficiently. Thinking of my future employment, and linking back to growing my portfolio, I must consider the programming languages that employers may want and brush up on them accordingly. Especially reflecting on my CPD journal, those which I have struggled with up to this point. My journal shows an increasing frustration with HTML over the weeks, a simple language that should be a tool in any programmer's belt. So, with this in mind, I grudgingly plan to sit the PluralSight course `HTML Fundamentals' by Matt Milner. I aim on being able to produce a better website for my next small project, than the atrocities that represent `Battle Screens' and `Paper AR Plane'. Other areas that I hope this will inspire me to refresh, include Python, and building LAMP servers.

\section*{Outreaching}

Due in part to many years in hospitality, I find it easy to strike up conversations with strangers. Creating rapport and finding a common ground is a skill that has been invaluable over the years. With the mantra ``it's not what you know, it's who you know'' ever present, and after being told that the UK game developing scene is very small and close-knit, I would like to start networking as early as possible. The fortnightly guest lecturers are prime candidates for this, and having contacted one recently; Andrew Fray, I realise are more than open to a few friendly emails back and forth. I gained valuable different advice and insight from our conversations, and have left the conversation open to be picked up again at any point in the future. I like this. This feels valuable. I regret not having started sooner. Before the end of this term I plan on contacting at least one more guest speaker. Be it a future one, or past, depends on how much I enjoy their lecture I suppose.

I have also started to visit local businesses. Firstly ``Buzz Interactive'' which advertises itself as `Web, Mobile apps, and Games', just by visiting unannounced to introduce myself. I did not particularly warm to their office, so I would like to continue visiting other local valid businesses, to grow my understanding of how these businesses operate, and to perhaps find one that feels like a good fit for myself. As an easy goal, I aim to visit at least one more before the end of term.

\section*{Learning or repeating?}

It has become relatively easy now having released two apps for mobile devices using Unreal Engine, to release more. This game engine has become a comfortable space for me. Whilst this can be productive, it would be a mistake in terms of my learning goals to stay comfortable. It is a weakness of mine that I do not want to move on to another engine like Unity. These three years at university are for me to learn and grow my portfolio, and that will not be achieved by repeatedly using the same software to make games for the same platform. Over the next year I will make the effort to move into the unfamiliar, and release a game made with Unity. That is, after maybe just one more Unreal Engine iOS game...

\section*{Conclusion}

I am still enormously enjoying my studies at Falmouth University, and while I believe that for the most part, my approach is the correct one, I must make the effort to continue and not lose momentum. Whilst still productive, it would be folly to stay in my comfort zone, and it is to this broader approach I must renew my efforts. Networking as early as possible takes effort but will be ultimately invaluable. After all, who wouldn't want to hear from a mature student with ideas far above his station?

\end{document}
\end{document}
